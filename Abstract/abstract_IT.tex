\documentclass[11pt,a4paper]{article}

\usepackage[italian]{babel}
\usepackage[utf8]{inputenc}
\usepackage[T1]{fontenc}
\usepackage{microtype}

\title{Tensor Train Decomposition\\A tensor representation format\\for high-dimensional problems}
\author{Tommaso Bianucci}
\date{9 giugno 2017}

\begin{document}
\maketitle
\begin{center}
%\section*{Tensor Train Decomposition\\A tensor representation format\\for high-dimensional problems}
\subsection*{Abstract}
\end{center}

Presentiamo un metodo di rappresentazione di tensori --~array multidimensionali~-- che è allo stesso tempo semplice, compatto e computazionalmente utilizzabile, chiamato \emph{Tensor Train Decomposition}.

Questo metodo non soffre inerentemente del \emph{curse of dimensionality} - non dipende quindi esponenzialmente dal numero di dimensioni - e il suo calcolo si basa su approssimazioni di rango basso di particolari matrici associate al tensore stesso (\emph{dispiegamenti} o \emph{unfolding matrices}).
Questo permette di sfruttare metodi solidi di approssimazione di matrici, come la \emph{Singular Values Decomposition}, per ottenere un algoritmo di approssimazione del tensore che dimostriamo essere stabile.

Nella tesi partiamo con una introduzione sui tensori, dando le definizioni, le proprietà di base e gli strumenti che ci serviranno in seguito.
Illustriamo poi velocemente le due decomposizioni classiche, Canonica e Tucker, assieme ai loro punti di forza e di debolezza.

Arriviamo quindi a descrivere il formato Tensor Train e mostrando un algoritmo di approssimazione, di cui dimostriamo la stabilità. Mostriamo che il formato permette di svolgere agevolmente operazioni di algebra lineare di base e illustriamo un algoritmo per l'arrotondamento di un tensore già in formato TT.

Inoltre mostriamo come il numero di parametri necessari per la rappresentazione non è esponenziale rispetto alla dimensionalità del tensore in ingresso e come invece dipenda dai ranghi scelti per le matrici di \emph{unfolding} durante la procedura di approssimazione.

Infine illustriamo una tecnica di approssimazione incrementale che non richiede di immagazzinare esplicitamente il tensore originario in memoria. Questo risulta necessario in scenari di elevata dimensionalità, in cui facilmente si raggiungono i limiti fisici di spazio del calcolatore.
Questo approccio, chiamato TT-Cross, sfrutta la possibilità di calcolare tramite una funzione gli elementi del tensore date le loro coordinate e facendolo solo per un opportuno sottoinsieme di elementi.

Concludiamo infine con una serie di esempi numerici che mostrano il basso numero di parametri necessari, rispetto alla rappresentazione esplicita, per immagazzinare l'approssimazione di alcuni tensori ottenuti valutando funzioni multidimensionali su una multi-grid. In particolare, illustriamo come si possano rappresentare agevolmente tensori a 100 o 200 dimensioni su una comune workstation.

\end{document}
