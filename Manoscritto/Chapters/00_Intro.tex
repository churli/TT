\chapter{Introduction}
Tensors are multidimensional arrays, that generalize matrices and vectors to higher dimension. More formally, a $d$-way or $d$th-order tensor is an element of the tensor product of $d$ vector spaces.

Tensors are naturally suited to represent multidimensional problems: that's why they commonly arise in many applied contexts. From scientific computing to quantum mechanics, from signal analysis to computational chemistry, there is a strong and wide interest in having robust, fast and memory efficient methods for handling tensors~\cite{tensorreview}.%\todo{Cerca un'applicazione specifica, tipo guarda se TT è in uso in metodi di risoluzione PDEs...}

In this work we will focus on the problem of representing a tensor, either exactly or through approximation, in a low-parametric format.
We will provide a quick overview of the basic definitions and tools and of two classic tensor decomposition. We will then present the \emph{Tensor Train Decomposition}, a recently developed low-parametric format for representing general tensors. We will see the decomposition main properties, the algorithms for compressing from the \emph{full-tensor} to the \emph{tensor-train}\footnote{or just \e{TT}} format and how to perform all the basic linear algebra operations in the compressed format. Eventually, chapter 5 covers the \emph{TT-cross} algorithm, presenting a way to incrementally build a tensor approximation by using a specific small subset of the tensor elements.

\paragraph{The curse of dimensionality.}
Problems involving a high number of dimensions are intrinsically difficult to handle: the computational resources, as memory or number of operations, needed for a $d$-dimensional problem, grow exponentially in $d$. Many real world applications involve a number of dimensions that can be as high as 10, 100 or even 1000, making them impossible to be handled using standard numerical methods.\\
Our aim is to show a tensor representation that:
\begin{itemize}
\item Does not inherently suffer from the curse of dimensionality,
\item Has stable algorithms to perform the compression step,
\end{itemize}
making it feasible to store high dimensional tensors and to perform computations on~them.
