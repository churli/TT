\chapter{Introduction}
Tensors are multidimensional arrays, that generalize matrices and vectors to higher dimension. More formally, an \N-way or \N th-order tensor is an element of the tensor product ot \N vector spaces.

Tensors are naturally suited to represent multdimensional problems: that's why they commonly arise in many applied contexts. From scientific computing to quantum mechanics, from signal analysis to computational chemistry, there is a strong and wide interest in having robust, fast and memory efficient methods for handling tensors.\todo{Refs here PLZ}

In this work we will focus on the problem of representing a tensor, either exactly or through approximation, in a low-parametric format.
We will provide a quick overview of the basic definitions and tools and of two classic tensor decomposition. We will then present the \emph{Tensor Train Decomposition}, a recently developed low-parametric format for representing general tensors. We will see the decomposition's main properties, the algorithms for compressing from the \emph{full-tensor format}, how to perform all the basic linear algebra operations in the compressed format. There will also be a chapter covering the \emph{TT-cross} algorithm, that gives a way to incrementally build a tensor approximation using only certain tensor elements.

\paragraph{The curse of dimensionality}
Problems involving an high number of dimensions are intrinsically difficult to handle: the computational resources, as memory or number of operations, needed for a $d$-dimensional problem grow exponentially in $d$. In many real world applications the number of dimensions can be as high as 10, 100 or even 1000, making them impossible to handle using the standard numerical methods.

Our aim is to present a tensor representation that:
\begin{itemize}
\item Does not inherently suffer from the curse of dimensionality
\item Has stable algorithms to perform the compression step \todo{devi accennare dei problemi di computazione della CP}
\end{itemize}
Making it feasible to store high dimensional tensors and to perform computations on them.
