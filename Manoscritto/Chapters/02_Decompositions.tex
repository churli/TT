\chapter{Decompositions}
wwwwww

\section{Canonical Decomposition (CANDECOMP/CP)}
The core idea behind the Canonical Decomposition\footnote{also called Parallel Factors, thus abbreviated as \emph{CP}} is to factorize a tensor into a sum of rank-one components. For example let \X an \N-th order tensor, we can factorize it as
\begin{equation}
  \X \approx \sum_{r = 1}^R a_r^{(1)} \circ a_r^{(2)} \circ \cdots \circ a_r^{(N)}
\end{equation}
where $R \in \NAT$ and $a^{(i)} \in \REAL^{I^i}$

\subsection{Tensor Rank}
Given a tensor \X we can define its \emph{rank} as \emph{the smallest number of rank-one tensors that generates \X as their sum}.

\section{Tucker decomposition}
wwwww
