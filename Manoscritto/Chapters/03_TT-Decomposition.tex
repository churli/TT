\chapter{Tensor-Train Decomposition}

The \emph{Tensor-Train decomposition} is the main topic of this work. It is a novel representation which was presented in this form by Oseledets in \cite{oseledets2011tt} although the format was already presented in a recursive form by the same author in \cite{oseledetstyrty2009}.

The idea behind the \emph{tensor train decomposition}, or simply \emph{TT-decomposition}, is to have a way of representing a higher order tensor using a sequence of third order tensors. This will prove to be useful, since we can use the \emph{Tucker} decomposition on these third order factors to further lower the number of parameters involved.

In this chapter we will go through some basic theoretical results, we will see how to compress a \e{full tensor} into the TT format and how to \e{round} a tensor approximation in the TT format. An algorithm to perform CP to TT conversion exists and is described in \cite{oseledets2011tt}, but it will not be discussed here as it is out of the purpose of this work.

We define the \emph{TT-decomposition} of a given tensor \A as follows:

\begin{Def}
  Let \A a $d$-dimensional tensor. The \emph{TT-decomposition} of \A will be given by a set of third order tensors $\G_1,\G_2,\dots,\G_d$ such as
  \begin{equation} \label{def:TT}
    \A(i_1,i_2,\dots,i_d) = G_1(i_1)G_2(i_2)\cdots G_d(i_d)
  \end{equation}
  Tensor $\G_1,\G_2,\dots,\G_d$ are called \emph{carriages} or \emph{cores} of the decomposition.
\end{Def}

Since, for each $k$, $\G_k$ is a third order tensor, each $G_k(i_k)$ is a matrix slice of it. Defining each $G_k(i_k)$ as a $r_{k-1} \times r_k$ matrix and given that our original tensor \A is $n_1 \times \cdots \times n_d$, we have that each carriage $\G_k$ can be treated as a $r_{k-1} \times n_k \times r_k$ tensor.

As we can see from \ref{def:TT}, the result is a $r_0 \times r_d$ matrix. Since we want to have a scalar result (i.e.\ the tensor element we are considering), we have to set $r_0 = r_d = 1$ as boundary condition.

The ranks $r_k$ are called \emph{compression ranks} or \emph{TT-ranks}. We will see how compression ranks can be computed and what their role is in aproximate representations.

We can also write the TT-decomposition of \A in elementwise form as:
\begin{equation} \label{def:TTindex}
  \A(i_1,\dots,i_d) = \sum_{\alpha_0,\dots,\alpha_d} \G_1(\alpha_0,i_1,\alpha_1) \cdots \G_d(\alpha_{d-1},i_d,\alpha_d)
\end{equation}

\section{Theoretical results}
We want to see now some fundamental properties of the TT-decomposition. Let \A a $d$-dimensional $n_1 \times \cdots \times n_d$ tensor. It is said to be in the TT format with cores $\G_k$ of size $r_{k-1} \times n_k \times r_k$ for $k=1,\ldots ,d$, with $r_0 = r_d = 0$, if its elements are defined by \ref{def:TT}.

\begin{Teo} \label{teo:ttsvd}
  Let \A a $d$-dimensional tensor and let, for each $k$, $A_k$ its standard $k$-th unfolding matrix defined in \ref{def:unfolding}.

  Then there exists a TT-decomposition with TT-ranks $r^{(tt)}_k$ such that $r^{(tt)}_k \leq r_k = rk(A_k) \; \forall k$.

  \begin{proof}
    Let \todo{Dimostrazione da rivedere (vedi \cite{tt-cross}).}
    $A_1$ the first unfolding matrix of \A. By the theorem's hypoteses $rk(A_1)~=~r_1$, therefore 
    % the SVD of $A_1$
    % \begin{equation*}
    %   A_1 = U \Sigma S^T = UV^T
    % \end{equation*}
    % where $V = \Sigma S$, will have $r_1$ singular values, 
    $A_1$ admits a dyadic representation $A_1 = UV^T$, with $U$ being a $n_1 \times r_1$ matrix and $V$ a $(n_2 \cdots n_d) \times r_1$\footnote{That such a representation exist can be easily proven by the use of SVD: $A_1 = U \Sigma S^T = UV^T$ where $V = S \Sigma^T$. Both $U$ and $V$ have $r_1 = rk(A_1)$ columns since $\Sigma$ contains exactly $r_1$ non-zero singular values},
    %
    which is expressed elementwise as
    \begin{equation*}
      A_1(i_1;i_2,\ldots,i_d) = \sum_{\alpha_1 = 1}^{r_1} U(i_1,\alpha_1) V(\alpha_1;i_2,\ldots,i_d)
    \end{equation*}
    We note that the matrix $U$ can also be seen as a $1 \times n_1 \times r_1$ tensor, which gives us the first core $G_1$ of the TT decomposition we are building.

    The matrix $V$ instead can be expressed as
    \begin{equation}\label{eq:Vdecomp}
      V = A_1^T U (U^T U)^{-1} = A_1^T W
    \end{equation}
    and elementwise as
    \begin{equation*}
      V(i_2,\ldots,i_d;\alpha_1) = \sum_{i_1 = 1}^{n_1} A_1^T(i_2,\ldots,i_d;i_1) W(i_1,\alpha_1)
    \end{equation*}
    We can now consider $V$ as a tensor indexed by $(\alpha_1 i_2)$ as one long index and by all the remaining $i_3, \ldots, i_d$, obtaining a $(d-1)$-dimensional tensor \V.

    The tensor \V has its own unfoldings, which will be denoted by $V_2,\ldots,V_d$. Assuming that $rk(V_k) \leq r_k$ holds, then we can apply the same procedure as above, so
    \begin{equation*}
      V(\alpha_1 i_2;i_3,\ldots,i_d) = \sum_{\alpha_2=1}^{r_2} G_2(\alpha_1 i_2;\alpha_2) V'(\alpha_2;i_3,\ldots,i_d)
    \end{equation*}
    This yields the next core tensor $G_2(\alpha_1,i_2,\alpha_2)$.
    We can now continue similarly by considering $V'$ as a $(d-2)$-dimensional tensor and so on, yielding all the cores up to $G_d(\alpha_{d-1},i_d)$ and therefore the entire TT-decomposition.

    We now have to prove the assumption $rk(V_k) \leq r_k$. Indeed, the $k$-th TT-rank of \A is equal to $r^{(tt)}_k \leq r_k$, therefore \A can be represented as\todo{Skeleton decomp.?}
    \begin{equation*}
      A(i_1,\ldots,i_d) = \sum_{\beta=1}^{r_k} F(i_1,\ldots,i_k,\beta) G(\beta,i_{k+1},\ldots,i_d)
    \end{equation*}
    Using this, together with \eqref{eq:Vdecomp}, we obtain
    \begin{align*}
      V_k &= V(\alpha_1 i_2,\ldots,i_k;i_{k+1},\ldots,i_d) \\
      &= \sum_{i_1 = 1}^{n_1} \sum_{\beta = 1}^{r_k} W(i_1, \alpha_1) F(i_1,\ldots,i_k,\beta) G(\beta,i_{k+1},\ldots,i_d) \\
      &= \sum_{\beta = 1}^{r_k} H(\alpha_1 i_2,\ldots,i_k,\beta) G(\beta,i_{k+1},\ldots,i_d) \\
    \end{align*}
    where
    \begin{equation*}
      H(\alpha_1 i_2,\ldots,i_k,\beta) = \sum_{i_1 = 1}^{n_1} W(i_1, \alpha_1) F(i_1,\ldots,i_k,\beta)
    \end{equation*}
    This separates row and column indices of $V_k$, yielding
    \begin{equation*}
      rank(V_k) \leq r_k
    \end{equation*}
  \end{proof}
  
\end{Teo}

The above proof of theorem \ref{teo:ttsvd} is constructive and yields an algorithm for computing the TT-decomposition of a given tensor using $d$ sequential SVDs on the auxiliary unfolding matrices.\footnote{The SVD is used to compute the dyadic decomposition of $A_k$.} This \e{full-to-TT} compression algorithm is called \e{TT-SVD} (\ref{algo:tt-svd}).

Since the SVD is used, we can leverage its properties: at each step, we can approximate $A_k$ with a lower-rank matrix via the SVD to have an approximate TT representation of the tensor. The following theorem lets us estimate the error that is introduced with this procedure, giving us a bound on the error of the TT approximation which depends on the error introduced by each SVD approximation on the unfoldings.

\begin{Teo} \label{teo:ttsvd_approx}
  Suppose that the unfoldings $A_k$ of the tensor \A are rank-$r_k$ only approximately, i.e.\ they satisfy

  \begin{equation} \label{teo:ttsvd_approx:ip:1}
  A_k = R_k + E_k, \quad rk(R_k) = r_k, \quad ||E_k||_F = \epsilon_k \quad:\quad k = 1,\ldots,d-1
  \end{equation}

  Then the TT-SVD yields a tensor \B in the TT format with TT-ranks $r_k$ and such that

  \begin{equation}
    ||\A - \B ||_F \le \sqrt{\sum_{k=1}^{d-1} \epsilon_k^2}
  \end{equation}
  
  \begin{proof}
    This proof proceeds by induction. For $d=2$ it follows directly from the properties of the SVD. Then we consider an arbitrary $d > 2$ and we assume that the property holds up to $d - 1$.  Now the first unfolding $A_1$ can be decomposed as
    \begin{equation*}
    A_1 = U_1 \Sigma V_1^T + E_1 = U_1 B_1 + E_1
    \end{equation*}
    where $U_1$ is $n_1 \times r_1$ and has orthonormal columns, $||E_1||_F = \epsilon_1$ and $B_1$ is naturally associated with a $(d-1)$-dimensional tensor $\B_1$ having elements $B_1(\alpha_1i_2,i_3,\ldots,i_d)$. This tensor $\B_1$ will be further decomposed in the TT-SVD algorithm: $B_1$ will be approximated by matrix $\hat{B}_1$.

    The global TT-SVD approximation error can be decomposed as
    \begin{equation*}
      \begin{split}
        ||\A-\B||_F^2 &= ||A_1 - U_1\hat{B}_1||_F^2 = ||A_1 - U_1(\hat{B}_1 + B_1 - B_1)||_F^2 \\
         &= ||A_1 - U_1 B_1||_F^2 + ||U_1(B_1 - \hat{B}_1)||_F^2
      \end{split}
    \end{equation*}
    and since, from the SVD definition, $U_1$ is orthonormal columns, we have
    \begin{equation} \label{teo:ttsvd_approx:proof:bound}
      ||\A - \B||_F^2 \le \eps_1^2 + ||B_1 - \hat{B}_1 ||_F^2
    \end{equation}

    The matrix $B_1$ is obtained from $A_1$ as
    \begin{equation*}
    B_1 = U_1^T A_1
    \end{equation*}

    If we now consider $B_{1,k}$ the $k$-th unfolding of the $(d-1)$-dimensional tensor $\B_1$, from the orthonormality of $U_1$, for each $k=2,\ldots,d-1$ we have\footnote{It's useful to recall here that the Frobenius norm is invariant under unfolding.}
    \begin{equation*}
      \begin{split}
        ||B_{1,k} - R_k||_F^2 &= ||U_1^T A_1 - R_k||_F^2 = ||A_k - R_k||_F^2\\
         &= \eps_k^2
      \end{split}
    \end{equation*}
    so the $B_{1,k}$ unfoldings satisfy \ref{teo:ttsvd_approx:ip:1} and by induction we have
    \begin{equation*}
      ||B_1 - \hat{B}_1 ||_F^2 \le \sum_{k=2}^{d-1} \eps_k^2
    \end{equation*}

    If we put this together with \ref{teo:ttsvd_approx:proof:bound} we have
    \begin{equation*}
      ||\A - \B||_F^2 \le \eps_1^2 + ||B_1 - \hat{B}_1 ||_F^2 = \sum_{k=1}^{d-1} \eps_k^2
    \end{equation*}
  \end{proof}
\end{Teo}

Two important corollaries follow from this theorem:
\begin{Cor}
  If \A admits a canonical approximation with $R$ terms %\todo{Che minchia è, una CP con R addendi?}
  and accuracy $\eps$, then there exists a TT-approximation of \A with TT-ranks $r_k < R$ and accuracy $\sqrt{d-1} \eps$.
\end{Cor}
\begin{Cor}
  Given a tensor \A and a set of rank bounds $r_k$, the best approximation\footnote{In the Frobenius norm} $\A^{best}$ of \A with TT-ranks bounded by $r_k$ always exists and the approximation $\B$ computed by the TT-SVD agorithm is \e{quasi-optimal}, i.e.
  \begin{equation*}
    ||\A - \B||_F \le \sqrt{d-1} ||\A - \A^{best}||_F
  \end{equation*}

  \begin{proof}
    Let
    \begin{equation*}
     \eps = \inf_{\tensor{C} \in TT(r_k)} ||\A - \tensor{C}||_F
    \end{equation*}
    where $TT(r_k)$ is the set of all the tensor trains with TT-ranks bounded by $r_k$. By definition of infimum there exists a sequence of tensor trains $\B^{(s)} : s \in \NN$ such that $lim_{s\to\infty} ||\A - \B||_F = \eps$.

    The elements of $\B^{(s)}$ are bounded and %\todo{Perchè averli bounded ci serve? BOLZANO-WEIERSTRASS!!!}
    there is some subsequence $\B^{(s_t)}$ converging elementwise to a tensor $\B^{min}$. Their unfolding matrices converge too. Now, for each $k$, the set of matrices having rank not greater than by $r_k$ is closed and since $\B^{(s_t)} \in TT(r_k)$
    \begin{equation*}
      rk(\B^{(s_t)}_k) \le r_k \quad \forall k
    \end{equation*}
    and thus
    \begin{equation*}
      rk(\B^{min}_k) \le r_k \quad \forall k
    \end{equation*}
    Moreover, $||\A - \B^{min}||_F = \eps$, so $\B^{min}$ is actually the best approximation $\A^{best}$.

    Finally, we note that $\eps_k \le \eps$, since each unfolding can be approximated at least with accuracy $\eps$, and the quasi-optimality bound follows directly from \ref{teo:ttsvd_approx}.
  \end{proof}
\end{Cor}

\begin{Cor} \label{cor:tt_tsvd_error}
  From \eqref{teo:ttsvd_approx} it also follows that if the SVD of the unfoldings are truncated at $\delta$, the approximation error of the whole TT will be $\delta \sqrt{d-1}$. So to obtain any prescribed accuracy $\eps$ on the TT whe have to set the treshold to $\delta = \frac{\eps}{d-1}$.
\end{Cor}

\begin{lstlisting}[float, caption=TT-SVD compression algorithm, label=algo:tt-svd, title=TT-SVD compression algorithm]
  ttSvd($A$,$\eps$):
    #Compute the truncation parameter $\delta$
    $\delta = \frac{\eps}{\sqrt{d-1}}||A||_F$
  
    $C = A$
    $r_0 = 1$
    for $k$ in range($1$,$d-1$) do
      #Unfold $C$
      $C = reshape(C,[r_{k-1}n_k,\frac{numel(C)}{r_{k-1}n_k}]$)
      #Compute the $\delta$-truncated SVD
      $C = USV + E$, where $||E||_F \leq \delta$
      $r_k = rank_{\delta}(C)$
      #Compute the new TT carriage
      $G_k = reshape(U,[r_{k-1},n_k,r_k])$
      $C = SV^T$
    done
    
    #Set last carriage
    $G_d = C$
    #Return the computed carriages
    return $G_1, G_2, \ldots, G_d$
\end{lstlisting}

\subsection{How many parameters?}
Let us take an $n_1 \times \cdots \times n_d$ tensor $\A \in I_1 \times \cdots \times I_d$. The number of parameters used to represent it in the full format is $n_1 \cdots n_d$. If we assume , $\forall k$, $n_k  = n$, we have an $n^d$ number of parameters used.

If \A is instead represented in the TT format, as in \ref{def:TTindex}, with TT-ranks $r_k$ we have that the number of parameters used is
\begin{equation*}
  \sum_{k=1}^d r_{k-1} n_k r_k
\end{equation*}
Since we know that $r_0 = r_d = 1$ and assuming that, $\forall k$, $r_k = r$, we have the following estimate
\begin{equation*}
  %n_1 r_1 + r_{d-1} n_d + \sum_{k=2}^{d-1} r_{k-1} n_k r_k =
  2nr + (d-2)nr^2
\end{equation*}

Using a Tucker decomposition on each of the three-dimesional TT-cores this can be lowered further to
\begin{equation*}
  dnr + (d-2)r^3
\end{equation*}

\section{Rounding in the TT format}
We have seen how a full tensor can be converted into the TT format using the TT-SVD algorithm (\ref{algo:tt-svd}). However it is not always possible, on real world machines, to have a full tensor to start with: the available memory can be insufficient to store all the elements and even computing all entries of the tensor can be a way too expensive task when dealing with high dimensionality.

It is thus interesting considering the case of having a tensor in an already structured format. More specifically, we can assume having a tensor in the TT format already, but stored with suboptimal ranks $r_k$. This is an important case since, as will be shown later, such tensors can result from performing basic linear algebra operations in TT format. In this case ranks $r_k$ have to be reduced while mantaining the prescribed accuracy: what we need here is a \e{rounding} algorithm, to recompress the tensor to optimal ranks while keeping the TT format.

\paragraph{Rounding technique}
Let \A in the TT format:
\begin{equation*}
  \A(1_1,\ldots,i_d) = G_1(i_1) \cdots G_d(i_d)
\end{equation*}
and assume its ranks $r_k$ are \e{increased} with respect to the optimal ranks $r_k^\prime$. We want now to estimate these optimal ranks $r_k^\prime \le r_k$ while keeping the prescribed accuracy $\eps$.

First, we try to compute $r_1^\prime$ and reduce our TT representation to this rank value. Our goal is to use the SVD on the unfoldings to discover the optimal ranks and to compute the new \e{recompressed} cores. A full description of the agorithm will be given at the end of this section (\ref{algo:tt-round}).

The first unfolding matrix $A_1$ can be written as
\begin{equation} \label{eq:unfolding_dyadic}
  A_1 = UV^T
\end{equation}
where
\begin{equation} \label{def:UV}
  \begin{split}
    U(i_1,\alpha_1) &= G_1(i_1,\alpha_1)\\
    V(i_2,\ldots,i_d;\alpha_1) &= G_2(\alpha_1,i_2) G_3(i_3) \cdots G_d(i_d)
  \end{split}
\end{equation}

We can compute the SVD of the unfolding $A_1$ making advantage of its representation \eqref{eq:unfolding_dyadic}. First we compute QR decompositions of $U$ and $V$
\begin{equation*}
  \begin{split}
    U &= Q_u R_u\\
    V &= Q_v R_v
  \end{split}
\end{equation*}
then we assemble a small $r \times r$ matrix
\begin{equation*}
  P = R_u R_v
\end{equation*}
and we compute its SVD
\begin{equation*}
  P = X \Sigma Y^T
\end{equation*}
where $\Sigma$ is an $\hat{r}\times\hat{r}$ diagonal matrix, $X$ and $Y$ are $r \times \hat{r}$ matrices with orthonormal columns and $\hat{r}$ is the $\eps$-rank of $P$. This gives that $\hat{r}$ is also the $\eps$-rank of $A_1$.

So we have that
\begin{equation*}
  \begin{split}
    \hat{U} &= Q_u X\\
    \hat{V} &= Q_v Y
  \end{split}
\end{equation*}
are matrices of dominant singular vectors of $A_1$.

\paragraph{Structured QR decomposition}
To be able to compute the SVD we thus have to go through the QR decompositions of the factor matrices $U$ and $V$.
The $U$ matrix is small, as we can see in \ref{def:UV}, so its QR decomposition can be computed directly with a standard algorithm.

However the $V$ matrix is very large and we will have to compute its QR in a smarter way. The idea is to incrementally compute the Q factor, keeping it in the TT format, while the small R factor will be stored explicitly. The tool we need to do this is the following lemma, showing that if the TT-cores satisfy certain othogonality properties, then the first unfolding will have orthogonal columns.
\begin{Lemma}
  Let a tensor \Z expressed as:
  \begin{equation*}
    \Z(\alpha_1,i_2,\ldots,i_d) = Q_2(i_2) \cdots Q_d(i_d)
  \end{equation*}
  where for each $k = 2,\ldots,d$, $Q_k(i_k)$ is an $r_{k-1} \times r_k$ matrix\footnote{For fixed $i_k$ with $k=2,\ldots,d$, the product is a $r_1$-long vector indexed by $\alpha_1$} and it satisfies the following orthogonality condition\footnote{By $I_s$ we denote the $s \times s$ identity matrix}
  \begin{equation} \label{eq:orth_cond}
    \sum_{i_k} Q_k(i_k) Q_k^{\top}(i_k) = I_{r_{k-1}}
  \end{equation}
  Then the $r_1 \times \prod_{k=2}^d$ first unfolding of \Z is a matrix $Z$ with orthonormal rows, i.e.
  \begin{equation*}
    (Z Z^T)_{\alpha_1,\hat{\alpha}_1} = \sum_{i_2,\ldots,i_d} Z(\alpha_1,i_2,\ldots,i_d) Z(\hat{\alpha}_1,i_2,\ldots,i_d) = \delta(\alpha_1,\hat{\alpha}_1)
  \end{equation*}

  \begin{proof}
    It follows immediately from the orthogonality of the cores:
    \begin{equation*}
      \begin{split}
        ZZ^T &= \sum_{i_2,\ldots,i_d} \Big( Q_2(i_2) \cdots Q_d(i_d) \Big) \Big( Q_2(i_2) \cdots Q_d(i_d) \Big)^T \\
        &= \sum_{i_2,\ldots,i_d} \Big( Q_2(i_2) \cdots Q_{d-1}(i_{d-1}) \Big)
        \underbrace{\Big( \sum_{i_d} Q_d(i_d) Q_d^{\top}(i_d) \Big)}_{I_{r_d}}
        \Big( Q_{d-1}^T(i_{d-1}) \cdots Q_2^T(i_2) \Big) \\
        &= \sum_{i_2,\ldots,i_d} \Big( Q_2(i_2) \cdots Q_{d-1}(i_{d-1}) \Big)
        \Big( Q_{d-1}^T(i_{d-1}) \cdots Q_2^T(i_2) \Big) \\
        &= \cdots = \sum_{i_2} Q_2(i_2) Q_2^{\top}(i_2) \\
        &= I_{r_1}
      \end{split}
    \end{equation*}
  \end{proof}
\end{Lemma}

Using the above lemma we can now design a fast algorithm for a structured QR decomposition of the matrix V, assuming it is in the TT format as defined in \ref{def:UV}. The idea is to perform a single right-to-left orthogonalization sweep through all the cores.

Let us start with $V$ written as
\begin{equation*}
  V(i_1,\ldots,i_d) = G_2(i_2) \cdots G_d(i_d)
\end{equation*}
Now, considering $G_d(i_d)$ as an $r_{d-1} \times n_d$ matrix (since $r_d = 1$) and orthogonalizing its rows, we can represent $G_d(i_d)$ as
\begin{equation*}
  G_d(i_d) = R_d Q_d(i_d)
\end{equation*}
where $Q_d(i_d)$ is an $r_{d-1} \times n_d$ matrix with orthonormal rows.
We now have
\begin{equation*}
  V(i_2,\ldots,i_d) = G_2(i_2) \cdots G^\prime_{d-1}(i_{d-1}) Q_d(i_d)
\end{equation*}
with
\begin{equation*}
  G^\prime_{d-1}(i_{d-1}) = G_{d-1}(i_{d-1}) R_d
\end{equation*}
This was the first step of the sweep. Suppose now that we already have a representation of the form
\begin{equation*}
  V(i_2,\ldots,i_d) = G_2(i_2) \cdots G_k(i_k) Q_{k+1}(i_{k+1}) \cdots Q_d(i_d)
\end{equation*}
where matrices $Q_s(i_s)$ satisfy \eqref{eq:orth_cond} for each $s = k+1, \ldots, d$. We now perform the orthogonalization step, defining
\begin{equation*}
  G^\prime_k(i_k) = R_k Q_k(i_k)
\end{equation*}
or, in index form
\begin{equation*}
  G^\prime_k (\alpha_{k-1},i_k,\alpha_k) = \sum_{\beta_k} R(\alpha_k,\beta_k) Q_k(\beta_{k-1},i_k,\alpha_k)
\end{equation*}
where $R_k$ is a matrix not depending on $i_k$ and holds
\begin{equation*}
  \sum_{i_k} Q_k(i_k) Q_k^T(i_k) = I_{r_k}
\end{equation*}
or, in index form
\begin{equation*}
  \sum_{i_k,\alpha_k} Q_k(\beta_{k-1},i_k,\alpha_k)Q_k(\hat{\beta}_k,i_k,\alpha_k) = \delta(\beta_k,\hat{\beta}_k)
\end{equation*}

Looking at the index forms of the above equations, we can see that $Q_k$ and $R_k$ can be computed by reshaping the $r_{k-1} \times n_k \times r_k$ tensor $G_k$ into a $r_{k-1} \times n_k r_k$ matrix and then orthogonalizing this matrix.

%\paragraph{Putting all the pieces together}

\todo{Aggiungere qualcosa sulle altre modes? (cfr. pag.2304 Oseledets)}

\begin{lstlisting}[float, caption=TT-rounding algorithm, label=algo:tt-round, title=TT-rounding algorithm]
  ttRound($A$,$\eps$):
    $G_1, \ldots, G_d =$ ttCores($\A$)
    #Compute the truncation parameter $\delta$
    $\delta = \frac{\eps}{\sqrt{d-1}}||A||_F$
    #Then perform right-to-left orthogonalization
    for $k$ in range($d$,$2$,step=$-1$) do
      $[G_k(\beta_{k-1;i_k\beta_k}), R(\alpha_{k-1};\beta_{k-1})] = QR_{rows}(G_k(\alpha_{k-1};i_k\beta_k))$
      $G_{k-1} = G_k \times_3 R$
    done
    #Compression of the orthogonalized representation
    for $k$ in range($1$, $d-1$) do
      #Compute $\delta$-truncated SVD
      $[G_k(\beta{k-1}i_k;\gamma_k),\Lambda,V(\beta_k,\gamma_k)] = SVD_\delta(G_k(\beta_{k-1}i_k;\beta_k))$
      $G_{k+1} = G_{k+1} \times_1 (V\Lambda)^T$
    done
    #Return the cores of the rounded TT
    return $G_1, G_2, \ldots, G_d$
\end{lstlisting}
%eof
