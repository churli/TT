\chapter{Tensor-Train Decomposition}

The \emph{Tensor-Train decomposition} is the main topic of this work. It was presented in this form by Oseledets in \cite{oseledets2011tt} although the format was already presented in a recursive form by the same author in \cite{oseledetstyrty2009}.

The idea behind the \emph{tensor train decomposition}, or simply \emph{TT-decomposition}, is to have a way of representing the higher order tensor using a sequence of third order tensors. This will prove to be useful, since we can use the \emph{Tucker} decomposition on these third order ``factors'' to further lower the number of parameters involved.

We define the \emph{TT-decomposition} of a given tensor \A as follows:

\begin{Def}
  Let \A a $d$-dimensional tensor. The \emph{TT-decomposition} of \A will be given by a set of third order tensors $G_1,G_2,\dots,G_d$ such as
  \begin{equation} \label{def:TT}
    A(i_1,i_2,\dots,i_d) = G_1(i_1)G_2(i_2)\cdots G_d(i_d)
  \end{equation}
  Tensor $G_1,G_2,\dots,G_d$ are called \emph{carriages} or \emph{cores} of the decomposition.
\end{Def}

Since, for each $k$, $G_k$ is a third order tensor, each $G_k(i_k)$ is a matrix slice of it. Defining each $G_k(i_k)$ as a $r_{k-1} \times r_k$ matrix and given that our original tensor \A is $n_1 \times \cdots \times n_d$, we have that each carriage $G_k$ can be treated as a $r_{k-1} \times n_k \times r_k$ tensor.

As we can see from \ref{def:TT}, the result is a $r_0 \times r_d$ matrix. Since we want to have a scalar result (i.e. the tensor element we are considering), we have to set $r_0 = r_d = 0$ as boundary condition.

The ranks $r_k$ are called \emph{compression ranks} or \emph{TT-ranks}. We will see how compression ranks can be computed and what their role is in aproximate representations.

We can now write the TT-decomposition of \A in the indexed form as:
\begin{equation} \label{def:TTindex}
  A(i_1,\dots,i_d) = \sum_{\alpha_0,\dots,\alpha_d} G_1(\alpha_0,i_1,\alpha_1) \cdots G_1(\alpha_{d-1},i_d,\alpha_d)
\end{equation}

\section{Theorical results}
We want to see now some fundamental properties of the TT-decomposition. Let \A a $d$-dimensional $n_1 \times \cdots \times n_d$ tensor. It is said to be in the TT format with cores $G_k$ of size $r_{k-1} \times n_k \times r_k$ for $k=1,\ldots ,d$, with $r_0 = r_d = 0$, if its elements are defined by \ref{def:TT}
