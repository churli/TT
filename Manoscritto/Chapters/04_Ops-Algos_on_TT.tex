\chapter{Linear algebra on TT: operations and algorithms}
Now we will see how some basic operations can be performed on the TT format and what is their complexity.

\section{Addition and multiplication by a number}
Let \A and \B $d$-dimensional tensors that can be written in the TT format as
\begin{equation*}
  \begin{split}
    \A &= A_1(i_1) \cdots A_d(i_d)\\
    \B &= B_1(i_1) \cdots B_d(i_d)
  \end{split}
\end{equation*}

\paragraph{Multiplication by a number}
This is a very easy operation, since to scale a tensor by $\alpha$ it is sufficient to scale one of the cores by it.

\paragraph{Addition}
The addition of these two tensors can be performed by \emph{merging} each couple of cores, i.e. the sum $\C = A + B$ will have \e{central} cores
\begin{equation*}
  C_k(i_k) = 
  \begin{pmatrix}
    A_k(i_k) & 0\\
    0 & B_k(i_k)
  \end{pmatrix}
\end{equation*}
and \e{border} cores
\begin{equation*}
  \begin{split}
    C_1(i_d) &= 
  \begin{pmatrix}
    A_1(i_1) & B1(i_1)
  \end{pmatrix}\\
  C_d(i_d) &= 
  \begin{pmatrix}
    A_d(i_d)\\
    B_d(i_d)
  \end{pmatrix}
  \end{split}
\end{equation*}

We can easily prove the above by direct multiplication:
\begin{equation*}
  C_1(i_1) \cdots C_d(i_d) = A_1(i_1) \cdots A_d(i_d) + B_1(i_1) \cdots B_d(i_d)
\end{equation*}

By the definition of the cores, we can see that summing tensors leads to a rank increase. For each core of the sum, we have a rank that is the sum of the ranks of the underlying cores. This makes the addition operation a good test bench for the rounding procedure.
