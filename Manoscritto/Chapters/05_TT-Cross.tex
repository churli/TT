\chapter{TT-Cross}
So far we have seen how a given tensor can be:
\begin{itemize}
\item compressed from the \e{full format} to the TT format
\item manipulated via various operations directly into the TT format
\item \e{rounded} (i.e. recompressed) without leaving the TT format
\end{itemize}

However, we still have a very relevant dependency on either the full representation format or the canonical decomposition as unique ways to input data to the TT format.
In fact, in case of a high value of $d$, we likely may be unable to represent the full tensor in order to perform the compression at the first instance. On the other hand, a canonical representation of tensors is usually already available only in few special cases (e.g. for discrete operators).

Fortunately, we can notice that, in all analytical problems and in many practical applications, tensors are often described implicitly by a procedure letting us compute any of its elements.
This is what is called a \e{black-box} tensor.

Having to deal with a black-box tensor can result in a relevant advantage: we can compute any of its element given their coordinates, thus we don't have to store the full multi-dimensional array of elements in order to be able to access them.

The question we want to answer in this chapter is: \e{Can we exploit this advantage? Can we find a way to have an arbitrary good approximation of the tensor just by using a limited subset of its element?}

\paragraph{TT-cross}
Here we are going to describe a low-rank tensor approximation techinique using a subset of few, well-chosen, elements.
The fundamental idea, described in \cite{tt-cross}, is to extend the dyadic (skeleton) decomposition for matrices to higher dimensions using the TT format.
The outcoming algorithm is called \e{TT cross} and it is shown \todo{da dimostrare? o forse è meglio mettere 'it turns out'?} that, if TT ranks are bounded by $r$, then only $\ord(dnr^2)$ carefully-chosen elements are required in order to completely recover the tensor.

