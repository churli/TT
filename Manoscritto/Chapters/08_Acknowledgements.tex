\chapter{Acknowledgements}
First of all I would like to thank \emph{you}, for having resisted and persevered in reading this thesis up to this point. It surely was a remarkable proof of will and perseverance.

I would like to thank Professor Dario Bini for the help and guidance he provided through the long process of writing this thesis. Without his patience and availability you wouldn't have had the chance of holding these few pages in your hands.

A big \emph{thank you} goes to all my fellows at PHC for the great hacking sessions we had together (yes, including those \emph{delicious cooking-hacking sessions!}). Most of what I know about how software and hardware work comes from having learnt it from you.

Also, I would like to tank all my colleagues at ION: I have learnt a lot from each one of you, the whole spectrum ranging from how to properly answer the phone in English, to how to deal with angry customers and even how to write code (almost) like a pro. You taught me a lot about how the real world out there looks like.

\begin{otherlanguage}{italian}
Vorrei ringraziare tutti i miei amici, vicini e lontani, per la grande pazienza che dimostrate nel sopportarmi e supportarmi ormai da diversi anni. Nei confronti di tutti voi sono inguaribilmente colpevole di latitanza e con tutti voi colgo l'occasione per scusarmi di questo.

Vorrei ringraziare i miei genitori e mia nonna Marisa, per aver aspettato con pazienza questo momento.
\end{otherlanguage}

\begin{otherlanguage}{dutch}
Ik wil ook mijn belgische familie bedanken: ik heb van jullie eerst geleerd hoe het wereld verschillend en interessant is buiten de bubbel waar ik gegroeid was. Nu mis ik jullie elke keer dat ik niet naar Kontich kan komen.
\end{otherlanguage}

\begin{otherlanguage}{italian}
Vorrei ringraziare anche te, Elia, per il tuo stimpanarmi con mirabile maestria e dedizione, da ormai più di dieci anni, con il tuo clarinetto. E non posso lamentarmi, la batteria era molto peggio.

Grazie Elisa: per la tua forza, per la tua pazienza, per le tue mille risorse, per la tua incrollabile e altrettanto inspiegbile fiducia, per aiutarmi costantemente, per la tua pazienza (ancora), per farmi essere sempre parte delle cose meravigliose che fai e per sopportare i resoconti di tutte quelle noiose che faccio io.
Senza di te, senza il tuo incoraggiamento e senza la tua calma, oggi non ci sarebbe nessuna tesi da leggere.

Grazie per tutte le avventure e disavventure che hai voluto condividere con me e grazie per le molte e molte altre che, spero, condivideremo in futuro.

Vorrei infine dedicare questo modesto lavoro a mio nonno Luciano, che mi ha insegnato come la curiosità e il desiderio di imparare siano i valori da custodire con più cura, da applicare con più dedizione e che danno più senso alla vita, indipendentemente dall'età. Anche se oggi non puoi essere qui a leggerle, sappi che queste quattro pagine sono anche e soprattutto per te.
\end{otherlanguage}