\documentclass[11pt,a4paper]{article}
%\documentclass[11pt,a4paper,draft]{book}

%%% Packages
\usepackage{Styles/reportstyle}
%\usepackage{Styles/unipitesi}
\usepackage{Styles/commands}

\usepackage{amsfonts}
%%%
\title{Immagini animate e Tensor Train.\\Compressione e suoi effetti su dati ``visibili''.}
\author{Tommaso Bianucci}
%%%
\begin{document}
\maketitle

\begin{abstract}
  La \emph{Tensor Train Decomposition} (\emph{TT}), è un formato di rappresentazione di tensori di recente sviluppo che ha come principale obiettivo quello di fornire una buona approssimazione del tensore iniziale utilizzando un numero ridotto di \emph{parametri}, analogamente a quanto la \emph{SVD} permette di fare con le matrici.

  %% In particolare, questo metodo di rappresentazione \emph{rompe} la diendenza esponenziale dal numero di dimensioni, rendendo finalmente possibili applicazioni con alta dimensionalità.
  %% Per questo possiamo riferirci alla TT come strumento di \emph{compressione} di dati, in quanto permette di manipolare dati dalla elevata dimensionalità con un'impronta in memoria relativamente ridotta.

  La qualità della rappresentazione è stata definita e dimostrata da risultati teorici, ma cosa questa comporti a livello qualitativo sui dati in ingresso resta difficile da immaginare, complicato inoltre dalla intrinseca difficoltà di rappresentazione di dati a più di 3 dimensioni.

  La compressione di immagini è un classico esempio che permette apprezzare immediatamente il risultato della SVD, così abbiamo utilizzato un approccio analogo con la TT, comprimendo brevi animazioni.
  Le immagini animate costituiscono infatti un facile esempio di dato in 4 dimensioni: larghezza, altezza, colore e numero di frame.
  Abbiamo quindi compresso le animazioni in formato TT, misurato il \emph{rapporto di compressione} e ricostruito le animazioni approssimate per un facile confronto visivo con l'originale. \todo{Rivedi, anche alla luce di quanto riutilizzato sotto in Obiettivi}
\end{abstract}

%\listoftodos
%\tableofcontents

%%% WRITE TEXT HERE
\section{Obiettivi}
La \emph{Tensor Train Decomposition}, o più brevemente \emph{TT}, è un formato di rappresentazione di tensori di recente sviluppo che ha come principale obiettivo quello di fornire una buona approssimazione del tensore iniziale utilizzando un numero ridotto di \emph{parametri}, analogamente a quanto la \emph{SVD} permette di fare con le matrici.
In particolare, questo metodo di rappresentazione \emph{rompe} la dipendenza esponenziale dal numero di dimensioni, rendendo finalmente possibili applicazioni con alta dimensionalità.
Per questo possiamo riferirci alla TT come strumento di \emph{compressione} di dati, in quanto permette di manipolare dati dalla elevata dimensionalità con un'impronta in memoria relativamente ridotta.

  La qualità della rappresentazione è stata definita e dimostrata da risultati teorici, ma cosa questa comporti a livello qualitativo sui dati in ingresso resta difficile da immaginare, complicato inoltre dalla intrinseca difficoltà di rappresentazione di dati a più di 3 dimensioni.

  La compressione di immagini è un classico esempio che permette, letteralmente, di visualizzare il risultato della SVD, apprezzandone intuitivamente le caratteristiche principali. Così abbiamo pensato di seguire un approccio analogo con la TT: comprimere immagini, per poterne visualizzare immediatamente il risultato.

\section{Approccio}
Un tensore rappresentato in formato TT consta di un insieme di tensori 3-dimensionali i cui elementi, opportunamente moltiplicati tra di loro, restituiscono tutti gli elementi, entro un certo grado di approssimazione, del tensore di partenza.
Questo lascia immediatamente intuire come i vantaggi del formato TT siano tanto maggiori quanto maggiore è la dimensionalità del tensore originario e anche come i benefici su tensori 3-dimensionali siano esigui se non assenti da un punto di vista di spazio utilizzato e sicuramente vanificati dalla complessità e dal tempo computazionale aggiuntivi che la rappresentazione in formato TT richiedono.

Con questa premessa abbiamo cercato una tipologia di dati che avesse 4 dimensioni e che fosse facilmente visualizzabile. La scelta è caduta sulle immagini animate.

Le sequenze animate si prestano naturalmente ad essere rappresentate in 4 dimensioni, infatti
\begin{itemize}
\item queste sono sicuramente un insieme ordinato di immagini (\emph{frames}),
\item ogni frame è composto da una matrice di \emph{pixel},
\item ogni pixel può emettere i 3 colori primari (rgb\footnote{\emph{RGB} = red-green-blue (rosso-verde-blu), i tre colori primari per la sintesi sottrattiva.}), ognuno caratterizzato da un proprio valore di luminosità.
\end{itemize}
Quindi otteniamo le 4 dimensioni su cui si strutturano i dati di una sequenza animata: numero di frame per identificare la posizione temporale, larghezza e altezza per identificare la posizione del pixel, colore.

\section{Implementazione e compromessi}
Per l'implementazione abbiamo utilizzato il seguente approccio:
\begin{itemize}
\item estrarre i singoli frame dalle animazioni GIF in ingresso e convertirli in formato PPM utilizzando il comando \rawcode{convert} dalla suite \href{http://www.imagemagick.org/}{Imagemagick},
\item x
\item x
\end{itemize}
la libreria \href{https://github.com/oseledets/ttpy}{ttpy} di Ivan Oseledets

\section{Risultati}
Lorem ipsum

\section{Conclusioni}
Lorem ipsum


%%%
\end{document}
